Elsinore is a time-looping adventure game set in the world of William Shakespeare's Hamlet.
Players take on the role of Ophelia, navigating their way through Castle Elsinore and observing the events of the tragedy unfold in real-time.
Throughout the drama, Ophelia can acquire and present pieces of knowledge (called hearsay in-game) to the other characters, influencing their beliefs and desires, changing the events that take place, and thus significantly affecting the final outcome.
Ophelia is trapped in a time loop, repeating the events of the play over and over.

Elsinore combines hand-authored content and narrative simulation.
Individual scenes are hand-written but scheduled using a logic-based simulation.
The game state models the mental states of characters in a temporal predicate logic, and hand-authored events are preconditioned on these mental states.
These events occur in real time in a 3D environment; thus the simulation must schedule these events to avoid temporal conflicts with rooms and NPCs.
The player's primary tool, presenting hearsay, changes these mental states, and accordingly, which events may occur.

The game's representation was chosen primarily to enable and assist the game experience.
One of the player's primary challenges is understanding how to manipulate the behavior of the NPCs to avoid a tragic outcome.
The NPC's mental states are exposed to the player.
This information aids the player in piecing together the mystery of the characters' behaviors and motivations, and how the player's actions affect the events and the endings.

The logic-based game state enables several AI techniques used to control the simulation, provide drama management, and support design tools.
For the simulation, the game uses propositional theorem proving to ensure important events are not suppressed by scheduling conflicts with minor events.
Events, by design, are non-interruptible once started.
Many major events can only happen at very specific times.
Therefore, if a minor event related to a sidequest were to be scheduled immediately before a major event and they both share a character or a room, the major event would be unable to occur.
However, in the cases where major events don't happen due to player action, we ideally want minor events to fill those gaps.
Therefore, our simulation schedules a minor event if, in addition to all typical preconditions, the game can prove that no conflicting major event could possibly happen within the duration of the minor event.
Given the logical preconditions, and postconditions, the duration, and the resources for every event, a hand-written propositional theorem prover can determine whether a minor event should be allowed to happen.

Our game also uses this logical model for minor drama management.
While exploring the castle, the player may engage in idle conversations with other NPCs.
These conversations cannot change the logical state, though, like the rest of game's dialogue, it is determined by logical state.
We can use this idle conversation to guide the player.
NPCs discuss relevant and topical information with the player, such as reactions to character deaths.
They also often inform the player about important scenes that the player did not witness or remind them of upcoming important events.

Finally, we are using AI to support our design workflow.
Currently, our design tools are capable of visualizing game traces and performing static analysis of game scripts.
We are in the process of developing solver-powered design tools for automatic QA and computational critics.
Exploiting the discrete, logical representation and event simulation, we are using off-the-shelf tools such as SMT solvers to reason about an abstracted version of the game.
This allows us to answer game-design-relevant questions such as event reachability, and will hopefully support more sophisticated queries as development progresses.

Information about the game can be found at \url{https://elsinore-game.com/}.